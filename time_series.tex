% !TeX spellcheck = en_US
\documentclass[10pt,titlepage,oneside,openany]{report}


\usepackage[utf8]{inputenc}

\usepackage{geometry}
\geometry{verbose,tmargin=3cm,bmargin=3cm,lmargin=2cm,rmargin=2cm}

%\usepackage{newtxtext}
\usepackage{enumitem}
\setenumerate[1]{leftmargin=3em}
\usepackage{amssymb}
\usepackage{amsmath}
\usepackage{amsthm}
\usepackage{amsopn}
\usepackage{graphicx}
\usepackage{subfig}
\usepackage{mathrsfs}
\usepackage{mdframed}
\usepackage{tikz-cd}
\usepackage{appendix}
\usepackage{esint}
\usepackage{mathrsfs}
\usepackage[normalem]{ulem}
\usepackage{thmtools, thm-restate}
\usepackage{hyperref}
\usepackage{placeins}

\usepackage{verbatim}
\usepackage{mathtools}
\usepackage{fullpage}
\usepackage{dsfont}
\usepackage{epstopdf}
\usepackage{physics}
\usepackage{setspace}
\usepackage{epstopdf}
\usepackage[T1]{fontenc}
\usepackage{graphicx}
\usepackage{scalerel,stackengine}

\usepackage{authblk}

\usepackage{soul}
\usepackage{marginnote}
\usepackage{todonotes}
\newcommand{\mytodo}[2][]{{%
		\let\marginpar\marginnote
		\reversemarginpar
		\renewcommand{\baselinestretch}{0.8}%
		\todo[#1]{#2}}}
\newcommand{\note}[2][]{\renewcommand{\baselinestretch}{0.8}\todo[#1]{#2}}


\newcommand\numberthis{\addtocounter{equation}{1}\tag{\theequation}}




\theoremstyle{remark}
\newtheorem{Remark}{Remark}[section]
\newtheorem{Example}{Example}[section]

\theoremstyle{plain}
\newtheorem{Theorem}{Theorem}[section]
\newtheorem{Lemma}{Lemma}[section]
\newtheorem{Corollary}{Corollary}[section]
\newtheorem{Definition}{Definition}[section]
\newtheorem{Proposition}{Proposition}[section]

\newtheorem{assump}{Assumption}
\newtheorem{assumpB}{Assumption}
\renewcommand\theassump{H\arabic{assump}}
\renewcommand\theassumpB{B\arabic{assumpB}}
\DeclareRobustCommand{\rchi}{{\mathpalette\irchi\relax}}
\newcommand{\irchi}[2]{\raisebox{\depth}{$#1\chi$}}


\newtheorem{assumptionex}{Assumption}
\newenvironment{assumption}
{\pushQED{\qed}\renewcommand{\qedsymbol}{$\bullet$}\assumptionex}
{\popQED\endassumptionex}
\newcommand{\doublehookrightarrow}{%
	\mathrel{\mathrlap{{\mspace{4mu}\lhook}}{\hookrightarrow}}
}

\setcounter{secnumdepth}{2}
\setcounter{tocdepth}{2}


\DeclareMathOperator{\sign}{sign}
\DeclareMathOperator{\erfc}{erfc}
\DeclareMathOperator{\diverg}{div}
\DeclareMathOperator{\Op}{Op}
\DeclareMathOperator{\Ima}{Im}
\DeclareMathOperator{\RePart}{Re}
\DeclareMathOperator{\diag}{diag}
\DeclareMathOperator{\supp}{supp}

\def \qed {\hfill \vrule height6pt width 6pt depth 0pt}
\newcommand{\al}{\alpha}
\newcommand{\R}{\mathbb{R}}
\newcommand{\N}{\mathbb{N}}
\newcommand{\Z}{\mathbb{Z}}
\newcommand{\C}{\mathbb{C}}
\newcommand{\Q}{\mathbb{Q}}
\newcommand{\p}{\partial}
\newcommand{\unity}{{1\!\!\!\:\mathrm{l}}}
\newcommand{\rightv}{\overset{\rightharpoonup}}
\newcommand{\harpoon}{\overset{\rightharpoonup}}
\let\vec\mathbf
\newcommand{\hsnorm}[1]{\norm{#1}_{\text{HS}}}
\newcommand{\opnorm}[1]{\norm{#1}_{\text{op}}}
\newcommand{\trnorm}[1]{\norm{#1}_{\Tr}}



\AtBeginEnvironment{subappendices}{%
	\chapter*{Appendix}
	\addcontentsline{toc}{chapter}{Appendices}
	\counterwithin{figure}{section}
	\counterwithin{table}{section}
}

\allowdisplaybreaks


\numberwithin{equation}{section}


\stackMath
\newcommand\reallywidehat[1]{
	\savestack{\tmpbox}{\stretchto{
			\scaleto{
				\scalerel*[\widthof{\ensuremath{#1}}]{\kern-.6pt\bigwedge\kern-.6pt}%
				{\rule[-\textheight/2]{1ex}{\textheight}}
			}{\textheight}
		}{0.5ex}}
	\stackon[1pt]{#1}{\tmpbox}
}

\usepackage[ngerman,english]{babel}
\usetikzlibrary{babel}

\newenvironment{abstractpage}
{\cleardoublepage\vspace*{\fill}\thispagestyle{empty}}
{\vfill\cleardoublepage}
\newenvironment{abstract0}[1]
{\bigskip\selectlanguage{#1}%
	\begin{center}\bfseries\abstractname\end{center}}
{\par\bigskip}





\renewcommand{\tilde}{\widetilde}
\renewcommand{\leq}{\leqslant}
\renewcommand{\geq}{\geqslant}
\DeclarePairedDelimiter\ceil{\lceil}{\rceil}

\def\baselinestretch{1.2}

\title{}


\author[1]{Matthew Liew\thanks{\href{mailto:mliew@mail.uni-mannheim.de}{\nolinkurl{mliew@mail.uni-mannheim.de} }}}


\linespread{1.5} 
%%%%%%%%%%%%%%%%%%%%%%%%%%%%%%%%%%%%%%%%%%%%%%%%%%%%%%%%%%%%%%%%%%%%%%%%%%%%%%%%%%%%%%%%%%%%%%%%
\begin{document}
	
	\pagenumbering{roman}
	% lets go for the title page, something like this should be okay
	\begin{titlepage}
		\vspace*{1cm}
		\begin{center}
			{\Large Theoretical Time-Series Notes}\\
			\vspace{0.25cm} 
			{For myself and other mathematicians}\\
			\vspace{6cm}
			{Matthew Liew}\\
			\vspace{5cm}
			{Jan 2023}
			
		\end{center}
	\end{titlepage} 
	
	\clearpage

	\pagenumbering{arabic}

\chapter{Introduction}\label{intro_thesis}



\section{Stationary and Autocorrelation}

\begin{Definition}
	The \textbf{autocovariance} $$\gamma (s,t) := \mathds{E} \left[(x_s - \mu_s) (x_t-\mu_t)\right],$$ and \textbf{cross-covar.} $$\rho_{x,y}(s,t):= \mathds{E}\left[(x_s - \mu_{xs})(y_t - \mu_{ts})\right].$$
\end{Definition}

\begin{Definition}
	The \textbf{autocorrelation function} (ACF) is defined as
	\[
		\rho_{x} (s,t)  := \frac{\gamma(s,t)}{\sqrt{\gamma(s,s)\gamma(t,t)}},
	\]
\end{Definition}

\textbf{Notation}: $\rho_{x}(t, t+h)  \equiv \rho(h)$.

\begin{Definition}[Strictly stationary]
	For all $k \in \N$, $t_k \in \N$ and $c_k \in \R$. Then the time series is strictly stationary if
		\[
		\mathds{P} \big(x_{t_1} \leq c_1, \dots x_{t_k} \leq c_k \big) = \mathds{P} \big(x_{t_1+h} \leq c_1, \dots x_{t_k+h} \leq c_k \big),
		\]
		for given probability measure $\mathds{P}$ and constant $h \in \N$.
\end{Definition}
\clearpage

\begin{Definition}[Weakly stationary]
	The time series is strictly stationary if
	\begin{enumerate}
		\item $\mu_t = \mathds{E}[x_t]$ is constant, and
		
		\item $\gamma(s,t)$ depends only on $|s-t|$. 
	\end{enumerate}
\end{Definition}


\begin{Lemma}
	Assume time series is weakly stationary. Using the notation $\gamma (t, t+h) \equiv \gamma (h)$, we have
	\begin{enumerate}
		\item $|\gamma (t)| \leq \gamma(0)$,
		
		\item $\gamma(h) = \gamma(-h)$.
	\end{enumerate}
\end{Lemma}

\begin{proof}
	\begin{enumerate}
		\item By Cauchy-Schwarz inequality $|\gamma (t, t+h)|^2 \leq \gamma(t,t) \gamma(t+h,t+h)$ by definition $|\gamma(h)|^2 \leq \gamma(0) \gamma(0)$.
		
		\item From definition of covariance we have
		\begin{align*} 
			\gamma(h) &= \gamma(t+h-t)\\
			& = \mathds{E}\left[(x_{t+h}-\mu)(x_t - \mu)\right]\\
			& = \mathds{E}\left[(x_t - \mu)(x_{t+h}-\mu)\right]\\
			& = \gamma(t-(t+h)) = \gamma(-h)
		\end{align*}
	\end{enumerate}
\end{proof}

\begin{Example}
	Suppose $w_t$ for all $t\in N$ is white noise process, i.e. iid $w_t \approx N(0, \sigma_w^2)$. Given two series
		\begin{align*}
			x_t &= w_t + w_{t-1},\\
			y_t &= w_t - w_{t-1}.
		\end{align*}
	Then for any $h\in \N$, we get the following for (cross-)covariance:
	\begin{enumerate}
		\item \begin{align*}
					\gamma_x(0) &= \mathds{E}[(w_t + w_{t-1})^2] =  \mathds{E}[w_t^2] + \mathds{E}[w_{t-1}^2] + 2\underbrace{ \mathds{E}[w_t w_{t-1}]}_{=0}\\
					 & = 2\sigma_w^2,
					 \\
					 \gamma_y(0) &= \mathds{E}[(w_t - w_{t-1})^2] =  \mathds{E}[w_t^2] + \mathds{E}[w_{t-1}^2] - 2\underbrace{ \mathds{E}[w_t w_{t-1}]}_{=0}\\
					 & = 2\sigma_w^2.
				\end{align*}
		\item 
			\begin{align*}
				\gamma_x (1) &= \mathds{E}\left[(w_t + w_{t-1}) (w_{t+1} + w_t) \right] \\
						& = \mathds{E}[w_tw_{t+1}] + \mathds{E}[w_t^2] + \mathds{E}[w_{t-1} w_{t+1}] + \mathds{E}[w_{t-1}w_t]\\
						& = \sigma_w = \gamma_x(-1),
						\\
				\gamma_y (1) &=  \mathds{E}[w_tw_{t+1}] - \mathds{E}[w_t^2] + \mathds{E}[w_{t-1} w_{t+1}] - \mathds{E}[w_{t-1}w_t]\\
					&  = - \sigma_w = \gamma_y(-1).
			\end{align*}
		
		\item $\gamma_{xy}(0) =  \mathds{E}[w_t^2]  -  \mathds{E}[w_{t-1}^2]  = 0$ and $\gamma_{xy} (1) = cov (x_{t+1}, y_t) = -\sigma_w^2$ and $\gamma_{xy} (-1) = cov (x_{t}, y_{t-1}) = -\sigma_w^2$
		
		\item ACF
		\begin{equation}
			\rho_{xy} (h) = 
			\begin{cases}
				0 & h=0,\\
				1/2 & h=1,\\
				-1/2 & h=-1,\\
				0 & |h| \geq 2
			\end{cases}
		\end{equation}
	Thus the joint time series is stationary.
	\end{enumerate}
\end{Example}

\begin{Definition}[Linear Process]
	$\{x_t\}$ is linear process if 
	\[
	x_t = \mu + \sum_{j=-\infty}^\infty \psi_j w_{t-j}, \quad \sum_{j=-\infty}^\infty |\psi_j|< \infty,
	\]
	where $w_t$ is white noise.
\end{Definition}

\begin{Definition}[Gaussian Process]
	$\{x_t\}$ is Gaussian process if the vector $x := (x_{t_1}, x_{t_2},\dots, x_{t_n})' \in \R^n$ has a multivariate normal distribution with density function
	\[
	f(x) = \frac{1}{(2\pi)^{n/2}} \det\big(\Gamma\big)^{-1/2} \exp \left(-\frac{1}{2} (x-\mu)' \Gamma^{-1} (x-\mu) \right),
	\]
	where $\mu \in \R^n$ and $\Gamma:= var (x) = \left\{ \gamma(t_i, t_j);\ i,j=1,\dots,n\right\}$.
\end{Definition}

%%%%%%%%%%%%%%%%%%%%%%%%%%%
%%%%%%			ARIMA
%%%%%%%%%%%%%%%%%%%%%%%%%%%
\chapter{ARIMA}
$ARMA(p,q)$: 
\[
x_t - \phi_1 x_{t-1} - \dots - \phi_p x_{t-p} = w_t + \theta_t w_{t-1} + \dots + \theta_q w_{t-q}.
\]

\section{Autoregressive Model}

\begin{Definition}[$AR(p)$]
	Let $w_t$ be white noise, $AR(p)$ is defined as 
	\[
	x_t - \phi_1 x_{t-1} - \cdots - \phi_p x_{t-p} = w_t.
	\]
\end{Definition}

\begin{Definition}[Autoregressive Operator]
	Let $B$ be time-lagged operator, i.e. $Bx_t = x_{t-1}$, the autoregressive operator is defined as
	\[
	\phi(B) = 1-\phi_1 B - \phi_2 B^2 - \cdots - \phi_p B^p.
	\]
\end{Definition}

\begin{Example}
	Assuming  $\phi_j = \phi$ for all $j \in \N$, from AR(1)
		\begin{align*}
			x_t & = \phi x_{t-1} + w_t = \phi(\phi x_{t-2} + w_{t-1}) + w_t\\
				  & =  \phi^2 x_{t-2}  + \phi w_{t-1} + w_t\\
				  & \vdots\\
				  & = \phi^k x_{t-k} + \sum_{j=0}^{k-1} \phi^j w_{t-j}.
		\end{align*}
	Thus, if $|\phi| < 1$ and $x_t$ is sationary then for large $k\to\infty$, we have
		\[
		x_t = \sum_{j=0}^\infty \phi^j w_{t-j}.
		\]
	Taking expectation value,
	\[
		\mathds{E}[x_t] = \sum_{j=0}^\infty	 \phi_j \mathds{E}[w_t] = 0.
	\]
	Moreover, for $h\geq $ the autocovriance function is
	\begin{align*}
		\gamma(h) &= \mathds{E} \left[\left(\sum_{j=0}^\infty \phi^j w_{t+h-j} \right) \left(\sum_{j=0}^\infty \phi^k w_{t-k} \right)\ \right]\\
			& = \mathds{E} \big[(w_{t+h-1}+ \phi w_{t+h-2} + \cdots \phi^h w_{t} + \phi^{h+1} w_{t-1} + \cdots ) (w_t + \phi w_{t-1}+ \cdots)\big]\\
			& = \sigma_w^2 \sum_{j=0} \phi^{h+j} \phi^j = \sigma_w^2 \phi^h \sum_{j=0} \phi^{2j}\\
			& = \frac{\sigma_w^2 \phi^h}{1-\phi^{2}}.
	\end{align*}
And ACF is
	\[
	\rho(h) = \frac{\gamma(h)}{\gamma(0)} = \phi^h.
	\]
As well as
	\[
	\rho(h)  =\phi \rho(h-1).
	\]
\end{Example}

\begin{Definition}
	The \textbf{moving average operator} is defined as
	\[
		\theta(B) : = 1 + \theta_1 B + \theta_2 B^2 + \cdots + \theta_q B^q.
	\]
\end{Definition}

\clearpage
%\bibliographystyle{alphadin}
%\bibliography{timeseries}
%\clearpage


\end{document}